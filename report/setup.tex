\usepackage[a4paper,top=30mm,bottom=30mm,left=30mm,right=30mm,headheight=15pt]{geometry}

\usepackage[LGR]{fontenc}
\usepackage[english,greek]{babel}
\usepackage{fontspec}
\usepackage{fancyhdr}
\usepackage{titlesec}
\usepackage{float}
\usepackage{parskip}
\usepackage{graphicx}
\usepackage[colorlinks=true, allcolors=black]{hyperref}
\usepackage[acronym,toc,nopostdot,nogroupskip]{glossaries}

\setmainfont{Ubuntu}
\setmonofont{Fira Code}

\pagestyle{fancy}

\setlength{\parskip}{10pt}
\setlength\parindent{0pt}

\setlocalecaption{greek}{figure}{Εικόνα}

\makeglossaries

% Commands
\newcommand{\thesistitle}{Μελέτη και δημιουργία 3D εικονικού κόσμου για διδασκαλία αλγορίθμων}
\newcommand{\thesisauthor}{Θεόδωρος Μπάτσικας}
\newcommand{\thesisauthorid}{Α.Μ. 1058113}

\newcommand*\hideheader{
    \fancyhead{}
    \renewcommand{\headrulewidth}{0pt}
}
\newcommand*\showheader{
    \fancyhead[L]{\slshape{\thesistitle}}
    \renewcommand{\headrulewidth}{0.4pt}
}
\newcommand*\modifiedsectiontitle{
    \titleformat{\section}[block]{\mbox{}\\\LARGE\bfseries}{Κεφάλαιο \thesection\\[1.5em]}{0pt}{\Huge}
    \titlespacing{\section}{0pt}{*3}{*10}
}
\newcommand*\defaultsectiontitle{
    \titleformat{\section}{\normalfont\Large\bfseries}{\thesection}{1em}{}
    \titlespacing{\section}{0pt}{3.5ex plus 1ex minus .2ex}{2.3ex plus .2ex}
}

% Acronyms
\newacronym{3d}{3D}{Three Dimensional}
\newacronym{lts}{LTS}{Long-Term Support}
\newacronym{unity_ecs}{ECS}{Entity Component System}
\newacronym{unity_dots}{DOTS}{Data-Oriented Technology Stack}
\newacronym{rest}{REST}{Representational State Transfer}
\newacronym{api}{API}{Application Programming Interface}
\newacronym{http}{HTTP}{Hypertext Transfer Protocol}
\newacronym{sql}{SQL}{Structured Query Language}
\newacronym{nosql}{NoSQL}{Not Only SQL}
\newacronym{rdbms}{RDBMS}{Relational Database Management System}
\newacronym{jit}{JIT}{Just-In-Time}
\newacronym{aws}{AWS}{Amazon Web Services}
\newacronym{rpc}{RPC}{Remote Procedure Call}

% Glossary
\newglossaryentry{cheat_code}{name=cheat code,description={Κωδικός που επιτρέπει πρόσβαση σε κρυφές λειτουργίες ενός παιχνιδιού}}
\newglossaryentry{context_menu}{name=context menu,description={Μενού περιβάλλοντος που εμφανίζεται όταν ο χρήστης κάνει δεξί κλικ πάνω σε ένα αντικείμενο}}
\newglossaryentry{stateless}{name=stateless,description={Χαρακτηριστικό του συστήματος που δεν αποθηκεύει κατάσταση}}
\newglossaryentry{docker_container}{name=Docker container,description={Εικονικό περιβάλλον που εκτελείται ανεξάρτητα από το λειτουργικό σύστημα του υπολογιστή}}
\newglossaryentry{docker_image}{name=Docker image,description={Εικονικό περιβάλλον που περιέχει όλα τα απαραίτητα εργαλεία για την εκτέλεση ενός προγράμματος}}
\newglossaryentry{js_runtime}{name=JavaScript runtime,description={Περιβάλλον εκτέλεσης JavaScript κώδικα}}
\newglossaryentry{prod_env}{name=production environment,description={Περιβάλλον εκτέλεσης που χρησιμοποιείται για την παραγωγή του τελικού προϊόντος}}
\newglossaryentry{framework}{name=framework,description={Σύνολο βιβλιοθηκών και εργαλείων που χρησιμοποιούνται για την ανάπτυξη εφαρμογών}}
\newglossaryentry{endpoint}{name=endpoint,description={Σημείο πρόσβασης σε ένα σύστημα}}
\newglossaryentry{bucket}{name=data storage bucket,description={Αποθηκευτικός χώρος στον οποίο μπορούν να αποθηκευτούν αρχεία}}
\newglossaryentry{data_pipeline}{name=data pipeline,description={Αγωγός ροής δεδομένων}}
\newglossaryentry{data_warehouse}{name=data warehouse,description={Αποθήκη μεγάλου όγκου δεδομένων}}
\newglossaryentry{server}{name=server,description={Διακομιστής}}
\newglossaryentry{time_trial}{name=time trial,description={Αγώνας κατά τον οποίο ο παίκτης προσπαθεί να ολοκληρώσει μία διαδικασία όσο πιο γρήγορα μπορεί}}
\newglossaryentry{null}{name=null,description={Τιμή που χρησιμοποιείται για να υποδηλώσει την απουσία τιμής}}
\newglossaryentry{multiplayer}{name=multiplayer,description={Παιχνίδι στο οποίο παίζουν περισσότεροι από ένας παίκτες}}
\newglossaryentry{unity_client}{name=client,description={Παίκτης που συνδέεται στο παιχνίδι και δεν είναι ο διακομιστής}}
\newglossaryentry{unity_host}{name=host,description={Παίκτης που διαχειρίζεται το παιχνίδι και είναι ο ιδιοκτήτης του κόσμου}}
\newglossaryentry{unity_server}{name=server,description={Διακομιστής που διαχειρίζεται την επικοινωνία μεταξύ των παικτών}}
