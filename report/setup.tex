\usepackage[a4paper,top=30mm,bottom=30mm,left=30mm,right=30mm,headheight=15pt]{geometry}

\usepackage[LGR]{fontenc}
\usepackage[english,greek]{babel}
\usepackage{fontspec}
\usepackage{fancyhdr}
\usepackage{titlesec}
\usepackage{float}
\usepackage{parskip}
\usepackage{graphicx}
\usepackage[colorlinks=true, allcolors=black]{hyperref}
\usepackage[acronym,toc]{glossaries}

\setmainfont{Ubuntu}

\pagestyle{fancy}

\setlength{\parskip}{10pt}
\setlength\parindent{0pt}

\makeglossaries

% Commands
\newcommand{\thesistitle}{Μελέτη και δημιουργία 3D εικονικού κόσμου για διδασκαλία αλγορίθμων}
\newcommand{\thesisauthor}{Θεόδωρος Μπάτσικας}
\newcommand{\thesisauthorid}{Α.Μ. 1058113}

\newcommand*\hideheader{
    \fancyhead{}
    \renewcommand{\headrulewidth}{0pt}
}
\newcommand*\showheader{
    \fancyhead[L]{\slshape{\thesistitle}}
    \renewcommand{\headrulewidth}{0.4pt}
}
\newcommand*\modifiedsectiontitle{
    \titleformat{\section}[block]{\mbox{}\\\LARGE\bfseries}{Κεφάλαιο \thesection\\[1.5em]}{0pt}{\Huge}
    \titlespacing{\section}{0pt}{*3}{*10}
}
\newcommand*\defaultsectiontitle{
    \titleformat{\section}{\normalfont\Large\bfseries}{\thesection}{1em}{}
    \titlespacing{\section}{0pt}{3.5ex plus 1ex minus .2ex}{2.3ex plus .2ex}
}

% Acronyms
\newacronym{3d}{3D}{Τρισδιάστατος χώρος}

% Glossary
\newglossaryentry{vw}{name=εικονικός κόσμος,description={Εικονικός κόσμος}}
