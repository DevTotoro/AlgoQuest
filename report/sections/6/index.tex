\hideheader
\section{Συμπεράσματα \& μελλοντικά βήματα}

\subsection{Συμπεράσματα}

Η χρήση του \acrshort{3d} εικονικού κόσμου για τη διδασκαλία αλγορίθμων αποδείχθηκε γενικά επιτυχημένη. Οι συμμετέχοντες, με μέσο όρο ηλικίας μεταξύ 19 και 31 ετών, βρήκαν την εμπειρία ευχάριστη και εύχρηστη, όπως υποδεικνύεται από το μέσο σκορ του \acrshort{sus} που ήταν 75,72 \%. Αυτό το σκορ δείχνει ότι το παιχνίδι πέτυχε να κάνει τη μάθηση πιο ελκυστική και ενδιαφέρουσα για τους χρήστες.

Ωστόσο, η έλλειψη σαφών οδηγιών για τους αλγόριθμους προκάλεσε δυσαρέσκεια σε αρκετούς συμμετέχοντες. Αυτοί οι συμμετέχοντες ανέφεραν ότι χρειάζονται καλύτερη καθοδήγηση και εκπαιδευτικά υλικά για να κατανοήσουν τη λειτουργία των αλγορίθμων πριν ξεκινήσουν να παίζουν. Αυτό το σημείο είναι κρίσιμο, καθώς η κατανόηση των βασικών αρχών των αλγορίθμων είναι απαραίτητη για την αποτελεσματική χρήση του εκπαιδευτικού εργαλείου.

Επιπλέον, παρατηρήθηκε ότι οι παίκτες στο Guided επίπεδο δυσκολίας προσπαθούσαν αμέσως να αλληλεπιδράσουν με τα κλειδωμένα κουτιά, παρά την ένδειξη ότι ήταν κλειδωμένα. Αυτό υποδεικνύει μια πιθανή σύγχυση που προκύπτει από το κίτρινο χρώμα των κουτιών, το οποίο τραβά την προσοχή των παικτών. Αυτό το εύρημα δείχνει την ανάγκη για αλλαγή στην οπτική σχεδίαση, ώστε να γίνεται πιο σαφές ότι τα κουτιά δεν είναι προσβάσιμα.

Η σύγκριση των χρόνων ολοκλήρωσης των αλγορίθμων έδειξε ότι ο αλγόριθμος Bubble Sort απαιτεί περισσότερο χρόνο για ολοκλήρωση σε σχέση με τον αλγόριθμο Selection Sort. Αυτό είναι λογικό, δεδομένου ότι ο Bubble Sort είναι ένας πιο χρονοβόρος αλγόριθμος λόγω του μεγαλύτερου αριθμού βημάτων που απαιτεί. Αυτή η διαφορά υποδεικνύει ότι οι μαθητές χρειάζονται περισσότερο χρόνο και καθοδήγηση για να κατανοήσουν και να εφαρμόσουν τον Bubble Sort. Οι εκπαιδευτικοί μπορούν να χρησιμοποιήσουν αυτή την πληροφορία για να αφιερώσουν περισσότερο χρόνο στην διδασκαλία του συγκεκριμένου αλγορίθμου.

Η επιτυχία στα επίπεδα Time Trial ήταν περίπου 50 \%, υποδεικνύοντας ότι οι μαθητές βρίσκουν τα επίπεδα προκλητικά, αλλά καταφέρνουν να τα ολοκληρώσουν με επιτυχία σε σημαντικό ποσοστό. Αυτό δείχνει ότι τα επίπεδα Time Trial είναι ισορροπημένα σε ό,τι αφορά την πρόκληση που προσφέρουν, ενώ παράλληλα παρέχουν στους μαθητές την ευκαιρία να βελτιώσουν τις δεξιότητές τους. Το γεγονός ότι οι συμμετέχοντες έπαιξαν το επίπεδο Time Trial περισσότερες από 3 φορές δείχνει μια προσπάθεια βελτίωσης και ανταγωνιστικό πνεύμα. Αυτό υποδηλώνει ότι οι μαθητές δεν εγκαταλείπουν εύκολα και προσπαθούν να πετύχουν καλύτερα αποτελέσματα, κάτι που είναι θετικό για την εκπαιδευτική τους πορεία.

% ========================================

\subsection{Μελλοντικά βήματα}

Για τη βελτίωση της εμπειρίας χρήστη, είναι απαραίτητο να ενσωματωθούν πιο λεπτομερείς και σαφείς οδηγίες για κάθε αλγόριθμο. Αυτό μπορεί να επιτευχθεί μέσω της δημιουργίας διαδραστικών οδηγών ή επιπέδων εκπαίδευσης που θα προσφέρουν μια πιο ολοκληρωμένη κατανόηση πριν οι παίκτες προχωρήσουν στα κανονικά επίπεδα του παιχνιδιού. Η παροχή λεπτομερών εξηγήσεων και παραδειγμάτων για κάθε αλγόριθμο θα βοηθήσει τους χρήστες να κατανοήσουν καλύτερα τις λειτουργίες και τις εφαρμογές τους.

Επίσης, η αλλαγή της οπτικής σχεδίασης των κλειδωμένων κουτιών στο Guided επίπεδο είναι απαραίτητη. Το κίτρινο χρώμα που τραβά την προσοχή των παικτών πρέπει να αλλάξει ή να προστεθούν επιπλέον οπτικές ενδείξεις που να εξηγούν ξεκάθαρα ότι τα κουτιά είναι κλειδωμένα και μη προσβάσιμα. Αυτή η αλλαγή θα μειώσει τη σύγχυση και θα διευκολύνει τη ροή του παιχνιδιού.

Η αναθεώρηση των επιπέδων δυσκολίας είναι επίσης σημαντική. Τα επίπεδα πρέπει να προσαρμοστούν καλύτερα στις ικανότητες των μαθητών, παρέχοντας περισσότερη προετοιμασία και υποστήριξη για τους πιο δύσκολους αλγόριθμους, όπως ο Bubble Sort. Αυτό θα μπορούσε να περιλαμβάνει επιπλέον εκπαιδευτικό υλικό, παραδείγματα και εξάσκηση για να βελτιωθεί η κατανόηση και η απόδοση των μαθητών.

Για να ενθαρρυνθεί η εξερεύνηση των αλγορίθμων, η διάταξη των επιλογών των αλγορίθμων στο παιχνίδι πρέπει να αναθεωρηθεί. Η τρέχουσα διάταξη φαίνεται να επηρεάζει τις αποφάσεις των χρηστών, και η αναδιάταξη των επιλογών μπορεί να ενθαρρύνει τους χρήστες να εξερευνήσουν και να χρησιμοποιήσουν όλους τους διαθέσιμους αλγορίθμους.

Τέλος, η υποστήριξη του ανταγωνιστικού πνεύματος και της προσωπικής βελτίωσης των μαθητών είναι κρίσιμη. Η δημιουργία πρόσθετων κινήτρων, όπως πίνακες κατάταξης ή βραβεία για τις καλύτερες επιδόσεις, μπορεί να ενισχύσει το ανταγωνιστικό πνεύμα και να προωθήσει την προσωπική βελτίωση. Αυτά τα μέτρα θα κάνουν την εμπειρία του παιχνιδιού πιο ενδιαφέρουσα και θα ενθαρρύνουν τους μαθητές να συνεχίσουν να προσπαθούν για καλύτερα αποτελέσματα.

Η ανάπτυξη μιας ισχυρής κεντρικής υποδομής \gls{backend} είναι κρίσιμη για τη διασφάλιση της ομαλής λειτουργίας της πλατφόρμας. Αυτή η υποδομή πρέπει να μπορεί να υποστηρίξει τη δυναμική φόρτωση αλγορίθμων, να παρέχει στατιστικά στοιχεία σε πραγματικό χρόνο και να επιτρέπει την εύκολη διαχείριση και παρακολούθηση της απόδοσης των μαθητών. Η κεντρική υποδομή πρέπει να είναι επεκτάσιμη και ευέλικτη, ώστε να μπορεί να προσαρμοστεί σε μελλοντικές ανάγκες και απαιτήσεις.

Η δυνατότητα δυναμικής φόρτωσης αλγορίθμων στην πλατφόρμα είναι απαραίτητη για τη συνεχή ανανέωση και την προσαρμογή του περιεχομένου στις ανάγκες των χρηστών. Αυτό θα επιτρέψει την εύκολη προσθήκη νέων αλγορίθμων και τη βελτίωση των υπαρχόντων, χωρίς την ανάγκη για εκτεταμένες αλλαγές στον κώδικα. Η δυναμική φόρτωση θα εξασφαλίσει ότι η πλατφόρμα παραμένει ενημερωμένη και σχετική με τις σύγχρονες εκπαιδευτικές απαιτήσεις.
