\phantomsection
\addcontentsline{toc}{section}{Περίληψη}
\section*{Περίληψη}

Η παρούσα εργασία επικεντρώνεται στη δημιουργία ενός \acrshort{3d} εικονικού κόσμου για τη διδασκαλία αλγορίθμων, με στόχο να βελτιώσει τη μαθησιακή εμπειρία μέσω της οπτικοποίησης και της διαδραστικότητας. Το διαδραστικό αυτό περιβάλλον επιτρέπει στους μαθητές να αλληλεπιδρούν με αλγορίθμους σε πραγματικό χρόνο, βοηθώντας τους να κατανοήσουν καλύτερα τις αφηρημένες έννοιες που συνήθως παρουσιάζονται στην επιστήμη των υπολογιστών. Στην εργασία παρουσιάζεται η θεωρία και οι μέθοδοι διδασκαλίας των αλγορίθμων. Στη συνέχεια, εξετάζονται οι δυνατότητες των \acrshort{3d} εικονικών κόσμων για εκπαιδευτικούς σκοπούς, περιλαμβάνοντας παραδείγματα επιτυχημένων εφαρμογών. Ακολουθεί η ανάλυση και ο σχεδιασμός του \acrshort{3d} εικονικού κόσμου, περιγράφοντας την τεχνική υποδομή και τις σχεδιαστικές αποφάσεις, όπως η χρήση της πλατφόρμας Unity και η ανάπτυξη του \acrshort{rest} \acrshort{api} Server για τη διαχείριση των δεδομένων και των στατιστικών. Η αξιολόγηση της αποτελεσματικότητας του εκπαιδευτικού εργαλείου πραγματοποιήθηκε μέσω εμπειρικής μελέτης, όπου συλλέχθηκαν δεδομένα από τη χρήση του \acrshort{3d} εικονικού κόσμου από μαθητές. Τα αποτελέσματα έδειξαν ότι η χρήση του \acrshort{3d} περιβάλλοντος βελτίωσε την κατανόηση των αλγορίθμων και έκανε τη μάθηση πιο ελκυστική και ενδιαφέρουσα.

\pagebreak
