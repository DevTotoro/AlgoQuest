\hideheader
\section{Εισαγωγή}

\subsection{Σκοπός της εργασίας}

Ο σκοπός της παρούσας εργασίας είναι η μελέτη και η δημιουργία ενός \acrshort{3d} εικονικού κόσμου για τη διδασκαλία αλγορίθμων. Η εργασία επικεντρώνεται στην ανάπτυξη ενός διαδραστικού περιβάλλοντος που θα ενισχύσει τη μαθησιακή εμπειρία των μαθητών μέσω της εικονικής αναπαράστασης αλγορίθμων. Αυτό το περιβάλλον θα επιτρέπει στους μαθητές να αλληλεπιδρούν με τους αλγορίθμους, να πειραματίζονται και να κατανοούν καλύτερα τις έννοιες της πληροφορικής, προσφέροντας έτσι μια πιο βιωματική και εμπλουτισμένη μαθησιακή διαδικασία.

Η διδασκαλία αλγορίθμων συχνά περιλαμβάνει αφηρημένες έννοιες και πολύπλοκες διαδικασίες, γεγονός που δυσκολεύει την κατανόηση από τους μαθητές μέσω των παραδοσιακών μεθόδων διδασκαλίας. Ο \acrshort{3d} εικονικός κόσμος επιτρέπει την οπτικοποίηση των αλγορίθμων, βοηθώντας τους μαθητές να κατανοήσουν τις αφηρημένες έννοιες μέσω της άμεσης αλληλεπίδρασης και της πρακτικής εφαρμογής. Οι μαθητές μπορούν να δουν τα αποτελέσματα των ενεργειών τους σε πραγματικό χρόνο, να πειραματιστούν με διαφορετικές προσεγγίσεις και να ενισχύσουν τη μαθησιακή τους εμπειρία.

Επιπλέον, το διαδραστικό στοιχείο κάνει τη μάθηση πιο ελκυστική και ενδιαφέρουσα, μετατρέποντας τη θεωρία σε πράξη. Το περιβάλλον προσαρμόζεται στις ανάγκες των μαθητών, επιτρέποντας την εξατομικευμένη μάθηση και την προσθήκη νέων αλγορίθμων και ασκήσεων από τους εκπαιδευτικούς. Αυτή η προσέγγιση καθιστά τη μάθηση των αλγορίθμων πιο προσβάσιμη, ευχάριστη και αποτελεσματική.

% ========================================

\subsection{Συνεισφορά}
Η συνεισφορά της παρούσας εργασίας εστιάζει σε δύο κύριους τομείς: την εκπαιδευτική τεχνολογία και την επιστήμη των υπολογιστών.

Πρώτον, η εργασία εισάγει έναν καινοτόμο τρόπο διδασκαλίας αλγορίθμων μέσω ενός \acrshort{3d} εικονικού κόσμου. Αυτό το περιβάλλον επιτρέπει στους μαθητές να αλληλεπιδρούν με τους αλγορίθμους σε πραγματικό χρόνο, ενισχύοντας την κατανόηση και την πρακτική εφαρμογή των εννοιών.

\pagebreak
\showheader

Δεύτερον, η εργασία παρέχει εμπειρικά δεδομένα και αξιολογήσεις για την αποτελεσματικότητα των εικονικών κόσμων στη διδασκαλία. Αυτά τα δεδομένα συμβάλλουν στην έρευνα για την εκπαιδευτική τεχνολογία και προσφέρουν χρήσιμες πληροφορίες για τη βελτίωση και την προσαρμογή τέτοιων εργαλείων.

Επιπλέον, η πλατφόρμα που αναπτύσσεται είναι ευέλικτη και επεκτάσιμη, επιτρέποντας στους εκπαιδευτικούς να προσθέτουν νέους αλγορίθμους και ασκήσεις, προσαρμόζοντας το εργαλείο στις ανάγκες των μαθητών. Συνολικά, η εργασία αυτή συμβάλλει στην προώθηση και την εξέλιξη της διδασκαλίας των αλγορίθμων μέσω της χρήσης καινοτόμων τεχνολογιών.

% ========================================

\subsection{Δομή της εργασίας}

Η εργασία είναι οργανωμένη σε έξι κύρια κεφάλαια, τα οποία καλύπτουν τη θεωρία, την υλοποίηση και την αξιολόγηση του \acrshort{3d} εικονικού κόσμου για τη διδασκαλία αλγορίθμων.

\begin{itemize}
    \item \textbf{Κεφάλαιο 2, Διδασκαλία Αλγορίθμων:} Αυτό το κεφάλαιο παρουσιάζει τις βασικές πτυχές της διδασκαλίας αλγορίθμων, συμπεριλαμβανομένων της θεωρίας, της ανάλυσης, της σχεδίασης και των κλασικών αλγορίθμων. Αναλύονται επίσης οι διάφορες μέθοδοι διδασκαλίας που χρησιμοποιούνται για την κατανόηση και την εφαρμογή των αλγορίθμων, θέτοντας τις θεωρητικές βάσεις για την ανάπτυξη του εκπαιδευτικού εργαλείου.

    \item \textbf{Κεφάλαιο 3, \acrshort{3d} Εικονικοί Κόσμοι και Διδασκαλία Πληροφορικής:} Εξετάζονται οι δυνατότητες των \acrshort{3d} εικονικών κόσμων για εκπαιδευτικούς σκοπούς, με έμφαση στη χρήση τους στη διδασκαλία της πληροφορικής. Παρουσιάζονται παραδείγματα επιτυχημένων εφαρμογών και οι τρόποι με τους οποίους οι εικονικοί κόσμοι μπορούν να βελτιώσουν τη μαθησιακή εμπειρία μέσω της αλληλεπίδρασης και της οπτικοποίησης.

    \item \textbf{Κεφάλαιο 4, Ανάλυση - Σχεδιασμός:} Περιγράφεται το περιβάλλον υλοποίησης και οι σχεδιαστικές αποφάσεις που ελήφθησαν κατά την ανάπτυξη του \acrshort{3d} εικονικού κόσμου. Αναλύεται η τεχνική υποδομή, οι απαιτούμενες λειτουργίες και τα χαρακτηριστικά του εργαλείου, καθώς και οι επιλογές σχεδιασμού που έγιναν για να εξυπηρετήσουν καλύτερα τους εκπαιδευτικούς στόχους.

    \item \textbf{Κεφάλαιο 5, Μελέτη - Αξιολόγηση:} Παρουσιάζονται η μεθοδολογία και τα αποτελέσματα της αξιολόγησης του εκπαιδευτικού εργαλείου μέσω εμπειρικής μελέτης. Αξιολογείται η αποτελεσματικότητα του \acrshort{3d} εικονικού κόσμου στην κατανόηση και την εκμάθηση αλγορίθμων από τους μαθητές, και αναλύονται τα δεδομένα που συλλέχθηκαν για να προσδιοριστούν τα πλεονεκτήματα και τα μειονεκτήματα της προσέγγισης.

    \item \textbf{Κεφάλαιο 6, Συμπεράσματα - Μελλοντικά Βήματα:} Συνοψίζονται τα κύρια ευρήματα της εργασίας και προτείνονται κατευθύνσεις για μελλοντική έρευνα και ανάπτυξη. Παρουσιάζονται οι δυνατότητες βελτίωσης του εκπαιδευτικού εργαλείου και οι προοπτικές για περαιτέρω επέκταση της χρήσης των \acrshort{3d} εικονικών κόσμων στην εκπαίδευση.
\end{itemize}

Αυτή η δομή επιδιώκει να παρέχει μια ολοκληρωμένη προσέγγιση στη δημιουργία, την αξιολόγηση και τη βελτίωση ενός καινοτόμου εκπαιδευτικού εργαλείου, συμβάλλοντας σημαντικά τόσο στην εκπαιδευτική τεχνολογία όσο και στην επιστήμη των υπολογιστών.
