\subsection{Μεθοδολογία}

Με την ολοκλήρωση των σταδίων υλοποίησης, ακολούθησε η αξιολόγηση του παιχνιδιού με τη συμμετοχή χρηστών. Ο στόχος αυτής της αξιολόγησης είναι η συλλογή στατιστικών δεδομένων από τους παίκτες, προκειμένου να αξιολογηθεί η αποτελεσματικότητα των μεθόδων που χρησιμοποιήθηκαν για τη διδασκαλία αλγορίθμων. Συγκεκριμένα, θα αναλυθούν οι επιδόσεις των παικτών, η κατανόηση των εννοιών και η πρόοδός τους μέσα στο παιχνίδι. Αυτά τα δεδομένα θα επιτρέψουν την εξαγωγή συμπερασμάτων για το πόσο καλά οι εκπαιδευτικές τεχνικές ενσωματώθηκαν στο παιχνίδι και πόσο επιτυχημένες ήταν στη βελτίωση των δεξιοτήτων των παικτών στον προγραμματισμό και την κατανόηση αλγορίθμων. Μέσα από αυτή τη διαδικασία, θα είναι δυνατό να εντοπιστούν τυχόν αδυναμίες και να προταθούν βελτιώσεις για μελλοντικές εκδόσεις του παιχνιδιού.

\subsubsection{Απαραίτητα συστήματα}

Για την ομαλή και ολοκληρωμένη λειτουργία του παιχνιδιού, απαιτείται η ενσωμάτωση πολλών συστημάτων πέρα από το ίδιο το παιχνίδι, όπως ένας \gls{server} για το \acrshort{rest} \acrshort{api} και μία βάση δεδομένων. Για την αρχική φάση της μελέτης, όλα αυτά τα συστήματα αποφασίστηκε να εγκατασταθούν σε έναν μόνο υπολογιστή. Με αυτόν τον τρόπο, όλοι οι συμμετέχοντες θα έχουν τη δυνατότητα να παίξουν χρησιμοποιώντας τον ίδιο υπολογιστή, εξασφαλίζοντας έτσι την απλότητα και την εύκολη παρακολούθηση της απόδοσης και των αποτελεσμάτων.

Όταν το παιχνίδι θα φτάσει σε κατάσταση που θα μπορούν να παίξουν διάφοροι παίκτες μέσω του διαδικτύου, θα χρειαστεί να μοιραστεί μέσω \gls{repo}. Αυτό σημαίνει ότι το παιχνίδι θα πρέπει να γίνει διαθέσιμο στο ευρύ κοινό, ώστε να μπορούν να το κατεβάσουν και να το εγκαταστήσουν οι χρήστες στον δικό τους υπολογιστή.

Παράλληλα, για την υποστήριξη πολλών χρηστών και τη συλλογή δεδομένων σε πραγματικό χρόνο, θα πρέπει να κατασκευαστεί ένα κοινό σύστημα \acrshort{rest} \acrshort{api} και μία βάση δεδομένων στο διαδίκτυο. Για να επιτευχθεί αυτό, μπορούν να χρησιμοποιηθούν πλατφόρμες όπως το \acrshort{aws}\cite{noauthor_cloud_nodate} ή το Google Cloud\cite{noauthor_cloud_nodate-1}. Αυτές οι πλατφόρμες προσφέρουν τις απαιτούμενες υποδομές για τη δημιουργία επεκτάσιμων και αξιόπιστων συστημάτων, που μπορούν να διαχειριστούν μεγάλο όγκο δεδομένων και να εξυπηρετήσουν πολλούς χρήστες ταυτόχρονα.

Με την υλοποίηση αυτών των συστημάτων στο διαδίκτυο, θα εξασφαλιστεί η επεκτασιμότητα και η προσβασιμότητα του παιχνιδιού σε ένα ευρύτερο κοινό. Αυτό θα επιτρέψει τη συλλογή περισσότερων δεδομένων από ποικίλους χρήστες, προσφέροντας έτσι καλύτερη και πιο ολοκληρωμένη αξιολόγηση των μεθόδων διδασκαλίας που χρησιμοποιήθηκαν. Επιπλέον, θα διευκολύνει την παρακολούθηση της απόδοσης των παικτών και τη βελτίωση του παιχνιδιού, με βάση τα δεδομένα και τα σχόλια που θα συλλεχθούν.

Τελικά, η μετάβαση από ένα τοπικό σύστημα σε ένα διαδικτυακό σύστημα θα αποτελέσει ένα κρίσιμο βήμα για την επιτυχία του παιχνιδιού και την αποτελεσματική αξιοποίηση των δεδομένων που θα συγκεντρωθούν για τη μελέτη.

% ========================================

\subsubsection{Διαδικασία μελέτης}

Για τη μελέτη αυτή, προσκλήθηκαν \textbf{31} άτομα να συμμετάσχουν, παίζοντας το παιχνίδι και συμπληρώνοντας ένα ερωτηματολόγιο που αφορούσε την εμπειρία τους. Κατά τη διάρκεια του παιχνιδιού, οι συμμετέχοντες έλαβαν σαφείς οδηγίες για την πλοήγηση στα μενού και τη χρήση των πλήκτρων κίνησης, ώστε να εξοικειωθούν γρήγορα με το περιβάλλον του παιχνιδιού και να μπορούν να το χειριστούν αποτελεσματικά.

Οι συμμετέχοντες είχαν την απόλυτη ελευθερία να επιλέξουν οποιονδήποτε αλγόριθμο ήθελαν να μελετήσουν, σε οποιοδήποτε επίπεδο δυσκολίας τους ενδιέφερε. Είχαν επίσης τη δυνατότητα να επαναλάβουν τις προσπάθειές τους όσες φορές επιθυμούσαν, χωρίς περιορισμούς. Αυτή η ελευθερία επιλογών τους επέτρεψε να προσεγγίσουν το παιχνίδι με τον δικό τους ρυθμό και να πειραματιστούν με διάφορους αλγόριθμους, ενισχύοντας έτσι τη μαθησιακή τους εμπειρία μέσω της δοκιμής και του λάθους.

Για τους συμμετέχοντες που δεν είχαν προηγούμενη εμπειρία με αλγόριθμους, δόθηκαν επιπλέον εξηγήσεις σχετικά με τη λογική και τη λειτουργία του αλγορίθμου που επέλεξαν να μελετήσουν. Αυτή η εισαγωγική καθοδήγηση παρείχε στους αρχάριους μια βασική κατανόηση των αλγορίθμων, βοηθώντας τους να ξεκινήσουν με μεγαλύτερη αυτοπεποίθηση. Στη συνέχεια, μπορούσαν να παίξουν το παιχνίδι και να εξερευνήσουν τους αλγόριθμους σε βάθος, προκειμένου να αξιολογηθεί αν κατάφεραν να κατανοήσουν τη λογική τους μέσα από τη διαδικασία του παιχνιδιού.

% ========================================

\subsubsection{Ερωτηματολόγιο}

Για την αξιολόγηση του παιχνιδιού και της εμπειρίας, συντάχθηκε ένα ερωτηματολόγιο το οποίο περιείχε γενικές ερωτήσεις, καθώς και ερωτήσεις για τη χρήση της μετρικής \acrfull{sus}\cite{noauthor_system_2024}.

\textbf{Δομή ερωτηματολογίου}

Το ερωτηματολόγιο που δόθηκε στους συμμετέχοντες περιελάμβανε μια σειρά από διαφορετικούς τύπους ερωτήσεων, σχεδιασμένων να καλύψουν διάφορες πτυχές της εμπειρίας τους με το παιχνίδι. Αρχικά, οι συμμετέχοντες κλήθηκαν να συμπληρώσουν τις δημογραφικές ερωτήσεις, οι οποίες περιελάμβαναν πληροφορίες όπως το φύλο και η ηλικία τους. Αυτές οι ερωτήσεις χρησίμευσαν για τη συλλογή βασικών δεδομένων σχετικά με το προφίλ των συμμετεχόντων.

Στη συνέχεια, οι συμμετέχοντες απάντησαν σε ερωτήσεις που αφορούσαν τις προηγούμενες γνώσεις τους. Αυτό το τμήμα του ερωτηματολογίου στόχευε να καταγράψει το επίπεδο εμπειρίας και τις δεξιότητες των συμμετεχόντων σε σχέση με το αντικείμενο του παιχνιδιού, π.χ. εάν είχαν προηγούμενη εμπειρία με παρόμοια παιχνίδια ή αλγόριθμους που χρησιμοποιήθηκαν.

Μετά από αυτές τις εισαγωγικές ερωτήσεις, οι συμμετέχοντες κλήθηκαν να περιγράψουν την εμπειρία τους από το παιχνίδι. Σε αυτό το μέρος, ζητήθηκε από τους συμμετέχοντες να επιλέξουν τις φορές που προσπάθησαν την κάθε δυσκολία και να αναφέρουν ποιους αλγόριθμους χρησιμοποίησαν κατά τη διάρκεια του παιχνιδιού. Αυτές οι πληροφορίες ήταν σημαντικές για να κατανοηθεί πώς οι διαφορετικές επιλογές και στρατηγικές επηρέασαν την εμπειρία τους.

Στο επόμενο βήμα, οι συμμετέχοντες συμπλήρωσαν τις βαθμολογίες για το μέρος \acrshort{sus} του ερωτηματολογίου. Οι δέκα ερωτήσεις του \acrshort{sus} τους ζητούσαν να βαθμολογήσουν την εμπειρία τους σε μια κλίμακα από το 1 (έντονη διαφωνία) έως το 5 (έντονη συμφωνία), επιτρέποντας την ποσοτική αξιολόγηση της χρηστικότητας του παιχνιδιού.

Τέλος, οι συμμετέχοντες κλήθηκαν να μιλήσουν για την γενική εμπειρία τους από το παιχνίδι και τη διαδικασία. Σε αυτό το μέρος, τους ζητήθηκε να εκφράσουν τις σκέψεις και τα συναισθήματά τους σχετικά με το παιχνίδι, να αναφέρουν εάν το βρήκαν ενδιαφέρον ή όχι, και να προτείνουν πιθανές βελτιώσεις. Αυτή η ποιοτική ανατροφοδότηση παρείχε πολύτιμες πληροφορίες για το πώς οι χρήστες αντιλήφθηκαν το παιχνίδι και τι είδους εμπειρία είχαν συνολικά.

Η δομή αυτή του ερωτηματολογίου, που περιλάμβανε δημογραφικές πληροφορίες, προηγούμενες γνώσεις, συγκεκριμένες εμπειρίες στο παιχνίδι, ποσοτική αξιολόγηση της χρηστικότητας μέσω του \acrshort{sus} και γενική ανατροφοδότηση, βοήθησε στη συλλογή ενός ολοκληρωμένου συνόλου δεδομένων. Αυτά τα δεδομένα είναι κρίσιμα για την ανάλυση και τη βελτίωση της εμπειρίας χρήστη, παρέχοντας σαφή κατεύθυνση για το πώς μπορεί να γίνει το παιχνίδι πιο προσιτό και απολαυστικό για τους χρήστες στο μέλλον.

\textbf{\acrfull{sus}}

Το \acrfull{sus} αποτελεί ένα αξιόπιστο εργαλείο για τη μέτρηση της χρηστικότητας ενός συστήματος ή προϊόντος, όπως λογισμικά, ιστοσελίδες και εφαρμογές. Αναπτύχθηκε από τον John Brooke το 1986 και έκτοτε έχει χρησιμοποιηθεί ευρέως για την αξιολόγηση της εμπειρίας χρήστη. Το \acrshort{sus} αποτελείται από ένα σύνολο δέκα ερωτήσεων που οι χρήστες απαντούν μετά τη χρήση του συστήματος. Κάθε ερώτηση βαθμολογείται σε μια κλίμακα από το 1 (έντονη διαφωνία) έως το 5 (έντονη συμφωνία), παρέχοντας μια συνολική εκτίμηση της χρηστικότητας του συστήματος\cite{noauthor_what_nodate}.

Οι ερωτήσεις του \acrshort{sus} είναι οι εξής:

\begin{itemize}
    \item Νομίζω ότι θα ήθελα να χρησιμοποιώ αυτό το σύστημα συχνά.
    \item Βρήκα το σύστημα περιττά περίπλοκο.
    \item Νομίζω ότι το σύστημα ήταν εύκολο στη χρήση.
    \item Νομίζω ότι θα χρειαζόταν τη βοήθεια ενός τεχνικού για να μπορέσω να χρησιμοποιήσω το σύστημα.
    \item Βρήκα τις διάφορες λειτουργίες του συστήματος καλά ενσωματωμένες.
    \item Νομίζω ότι υπήρχαν πάρα πολλές ασυνέπειες στο σύστημα.
    \item Θα φανταζόμουν ότι οι περισσότεροι άνθρωποι θα μάθαιναν να χρησιμοποιούν αυτό το σύστημα πολύ γρήγορα.
    \item Βρήκα το σύστημα πολύ δύσχρηστο.
    \item Αισθάνθηκα σίγουρος/η χρησιμοποιώντας το σύστημα.
    \item Χρειάστηκε να μάθω πολλά πριν μπορέσω να ξεκινήσω να χρησιμοποιώ το σύστημα.
\end{itemize}

Για να υπολογιστεί το συνολικό σκορ του \acrshort{sus}, οι χρήστες καλούνται να απαντήσουν στις δέκα αυτές ερωτήσεις, κάθε μία βαθμολογούμενη από το 1 (έντονη διαφωνία) έως το 5 (έντονη συμφωνία). Οι θετικές ερωτήσεις (1, 3, 5, 7, 9) βαθμολογούνται αφαιρώντας το 1 από τη βαθμολογία τους, ενώ για τις αρνητικές ερωτήσεις (2, 4, 6, 8, 10) η βαθμολογία αφαιρείται από το 5. Στη συνέχεια, προστίθενται οι προσαρμοσμένες βαθμολογίες όλων των ερωτήσεων και το άθροισμα αυτό πολλαπλασιάζεται επί 2,5 για να προκύψει το τελικό σκορ, το οποίο κυμαίνεται από 0 έως 100\cite{brooke_sus_1995}.

Το τελικό σκορ \acrshort{sus} επιτρέπει την αξιολόγηση της χρηστικότητας του συστήματος. Ένα σκορ πάνω από 68 θεωρείται πάνω από το μέσο όρο, ενώ σκορ μεταξύ 70 και 80 υποδηλώνει καλή χρηστικότητα. Σκορ πάνω από 80 σημαίνει ότι το σύστημα έχει πολύ καλή χρηστικότητα. Αντίθετα, σκορ κάτω από 50 δείχνει ότι υπάρχει ανάγκη για σημαντικές βελτιώσεις στη χρηστικότητα. Αυτή η ποσοτική αξιολόγηση βοηθά στην αντικειμενική σύγκριση της εμπειρίας χρήστη μεταξύ διαφορετικών συστημάτων και μελετών\cite{noauthor_system_nodate}.

Το σκορ \acrshort{sus} μπορεί να αξιολογηθεί πιο εύκολα με τη χρήση μίας κλίμακας βαθμών \textbf{A, B, C, D, F} όπου οι βαθμοί αντιστοιχούν σε συγκεκριμένα εύρη σκορ, όπως φαίνεται στον πίνακα \ref{tab:sus_grades}\cite{bellio_sus_2023}.

\begin{table}[H]
    \centering
    \begin{tabular}{|c|c|}
        \hline
        \textbf{Βαθμός} & \textbf{Εύρος Σκορ} \\
        \hline
        A & 85-100 \\
        B & 70-84 \\
        C & 50-69 \\
        D & 35-49 \\
        F & 0-34 \\
        \hline
    \end{tabular}
    \caption{Κλίμακα Βαθμολόγησης \acrshort{sus}}
    \label{tab:sus_grades}
\end{table}

Η αξία του \acrshort{sus} έγκειται στη δυνατότητά του να παρέχει γρήγορα και εύκολα μια εκτίμηση της χρηστικότητας, με την προσέγγιση του να είναι ιδιαίτερα χρήσιμη σε αρχικά στάδια ανάπτυξης προϊόντων ή σε περιπτώσεις όπου απαιτούνται συγκριτικές αξιολογήσεις. Η απλότητα της κλίμακας και η ικανότητά της να καλύπτει ευρύ φάσμα παραμέτρων της χρηστικότητας το καθιστούν ένα πολύτιμο εργαλείο για τους σχεδιαστές και τους ερευνητές εμπειρίας χρήστη.

Επιπλέον, τα αποτελέσματα του \acrshort{sus} μπορούν να συνδυαστούν με ποιοτικές παρατηρήσεις και άλλα ποσοτικά δεδομένα για μια πιο ολοκληρωμένη εικόνα της εμπειρίας χρήστη. Αν και το \acrshort{sus} δεν αντικαθιστά άλλες μεθόδους αξιολόγησης, αποτελεί μια σημαντική προσθήκη στη \("\)φαρέτρα\("\) των εργαλείων χρηστικότητας, παρέχοντας γρήγορα και αξιόπιστα αποτελέσματα που μπορούν να καθοδηγήσουν βελτιώσεις και να υποστηρίξουν τη λήψη αποφάσεων στη διαδικασία ανάπτυξης.
